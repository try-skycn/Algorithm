\documentclass[12pt,a4]{article}

\usepackage{graphicx,amsmath,amssymb,amsthm, boxedminipage, bm}
\usepackage{hyperref}

\newtheorem{theorem}{Theorem}%[section]
\newtheorem{proposition}[theorem]{Proposition}
\newtheorem{lemma}[theorem]{Lemma}
\newtheorem{corollary}[theorem]{Corollary}
\newtheorem{definition}[theorem]{Definition}

\newcommand{\scalar}[2]{\ensuremath{\langle #1, #2\rangle}}
\newcommand{\floor}[1]{\left\lfloor #1 \right\rfloor}
\newcommand{\ceil}[1]{\left\lceil #1 \right\rceil}
\newcommand{\norm}[1]{\|#1\|}
\newcommand{\pfrac}[2]{\left(\frac{#1}{#2}\right)}
\newcommand{\nth}[1]{#1^{\textsuperscript{th}}}


\newcommand{\N}{\mathbb{N}}

\newcommand{\R}{\mathbb{R}}

\newcommand{\Exercise}[1]{\noindent\textbf{Exercise }\textbf{#1}}
\newcommand{\Question}[1]{\noindent\textbf{Question }\textbf{#1}}
\newcommand{\solution}{\\[8pt]\noindent\textit{Sol. }}

\newcommand\question[1]{\vspace{.25in}\hrule\vspace{.5em}\textbf{#1}\vspace{.5em}\hrule\vspace{.10in}}
\newcounter{exercise}

\newcommand{\exercise}{\addtocounter{exercise}{1}\noindent\textbf{Exercise \arabic{section}.\arabic{exercise}. }}

\begin{document}

\date{2015-12-08}

\author{Dumb ways to die\\
Runzhe Yang, Songyu Ke, Xingyuan Sun \& Tianyao Chen}

\title{Algorithm Analysis\\
  Homework Assignment 5 \\
}
\maketitle
\hrule\hrule\hrule\hrule
\setcounter{section}{4}
\section{Basic Probability Theory}
\setcounter{subsection}{0}

\begin{exercise}
	
\end{exercise}

\begin{exercise}

\end{exercise}

\begin{exercise}
	
\end{exercise}

\begin{exercise}
	\begin{enumerate}
		\item
		\item
		
			Firstly, for any distinct $u, v, w \in \mathbb{F}_{p}$
			\begin{eqnarray*}
				&&\text{Pr}[Y_u = i, Y_v = j, Y_w = k] \\
				= &&\text{Pr}[a + bu + cu^2 = i, a + bv + cv^2 = j, a + bw + cw^2 = k] \\
				= &&\text{Pr}[
				\left[
				\begin{matrix}
					1 & u & u^2 \\
					1 & v & v^2 \\
					1 & w & w^2
				\end{matrix}
				\right]
				\left[
				\begin{matrix}
					a \\
					b \\
					c
				\end{matrix}
				\right]
				=
				\left[
				\begin{matrix}
					i \\
					j \\
					k
				\end{matrix}
				\right]
				]
			\end{eqnarray*}
			
			Note that
			\[
				\det
				\left(\left|
				\begin{matrix}
					1 & u & u^2 \\
					1 & v & v^2 \\
					1 & w & w^2
				\end{matrix}
				\right|\right)
				=
				(v - u)(w - u)(w - v)
				\neq
				0
			\]
			
			So the system of linear equations above has and only has one solution in $\mathbb{F}^{3}_{p}$.
			Denote it as $(a_0, b_0, c_0)$.
			
			Therefore,
			\begin{eqnarray*}
				\text{Pr}[Y_u = i, Y_v = j, Y_w = k]
				= \text{Pr}[a = a_0, b = b_0, c = c_0]
				= \frac{1}{p^3}
			\end{eqnarray*}
			
			Secondly,
			\begin{eqnarray*}
				&&\text{Pr}[Y_u = i] \\
				= &&\text{Pr}[a + bu + cu^2 = i]
			\end{eqnarray*}
			
			Obviously, for any $a_0, b_0 \in \mathbb{F}_{p}$, there exists and only exists one $c_0 \in \mathbb{F}_{p}$, such that $a_0 + b_0 u + c_0 u^2 = i$.
			
			Thus, there are altogether $p^2$ solutions.
			
			So
			\begin{eqnarray*}
				&&\text{Pr}[Y_u = i] \\
				= &&\text{Pr}[a + bu + cu^2 = i] \\
				= &&\frac{p^2}{p^3} \\
				= &&\frac{1}{p}
			\end{eqnarray*}
			
			Similarly,
			\begin{eqnarray*}
				&&\text{Pr}[Y_v = j] \\
				= &&\text{Pr}[Y_w = k] \\
				= &&\frac{1}{p}
			\end{eqnarray*}
			
			Therefore,
			\[
				\text{Pr}[Y_u = i] \cdot \text{Pr}[Y_v = j] \cdot \text{Pr}[Y_w = k] = \frac{1}{p} \cdot \frac{1}{p} \cdot \frac{1}{p} = \frac{1}{p^3}
			\]
			
			Finally,
			\[
				\text{Pr}[Y_u = i, Y_v = j, Y_w = k] = \text{Pr}[Y_u = i] \cdot \text{Pr}[Y_v = j] \cdot \text{Pr}[Y_w = k]
			\]
			
			i.e., $Y_u$ are $3$-wise independent.
		\item
		
			$\mathcal{H} = \{ f | f(x) = a x^2 + b x + c \quad a,b,c \in \mathbb{F}_{p} \}$
			
			\begin{proof}
				Obviously, $|\mathcal{H}| = p^3$.
				
				For any distinct $u, v, w \in \mathbb{F}_{p}$ and arbitrary $i, j, k \in \mathbb{F}_{p}$,
				
				\begin{eqnarray*}
					&&\mathop{\text{Pr}} \limits_{h \in \mathcal{H}} [h(u) = i, h(v) = j, h(w) = k] \\
					= &&\mathop{\text{Pr}} \limits_{h \in \mathcal{H}} [a + bu + cu^2 = i, a + bv + cv^2 = j, a + bw + cw^2 = k] \\
					= &&\mathop{\text{Pr}} \limits_{h \in \mathcal{H}}[
					\left[
					\begin{matrix}
						1 & u & u^2 \\
						1 & v & v^2 \\
						1 & w & w^2
					\end{matrix}
					\right]
					\left[
					\begin{matrix}
						a \\
						b \\
						c
					\end{matrix}
					\right]
					=
					\left[
					\begin{matrix}
						i \\
						j \\
						k
					\end{matrix}
					\right]
					]
				\end{eqnarray*}
				
				Note that
				\[
					\det
					\left(\left|
					\begin{matrix}
						1 & u & u^2 \\
						1 & v & v^2 \\
						1 & w & w^2
					\end{matrix}
					\right|\right)
					=
					(v - u)(w - u)(w - v)
					\neq
					0
				\]
				
				So the system of linear equations above has and only has one solution in $\mathbb{F}^{3}_{p}$.
				Denote it as $(a_0, b_0, c_0)$.
				
				Therefore,
				\begin{eqnarray*}
					\mathop{\text{Pr}} \limits_{h \in \mathcal{H}}[h(u) = i, h(v) = j, h(w) = k]
					= \mathop{\text{Pr}} \limits_{h \in \mathcal{H}}[a = a_0, b = b_0, c = c_0]
					= \frac{1}{p^3}
				\end{eqnarray*}
			\end{proof}
		
	\end{enumerate}
\end{exercise}

\end{document}
