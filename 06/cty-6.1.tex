Suppose our algorithm $f$ has $s$ bits of memory. Then we have $S:|S|=2^s$ different states.

For any given sequence $\mathbf{a}$, we make $D(\mathbf{a})$ stands for the set
\[
	D(\mathbf{a}) = \{a_0,a_1,\ldots,a_{\mathrm{length}(\mathbf{a})-1}\}
\]
and $f(\mathbf{a})\in S$ stands for the state our program reach when we have received the sequence $\mathbf{a}$.

If we have two sequences $\mathbf{a}$ and $\mathbf{b}$, and they consists of different set of elements
\[
	D(\mathbf{a})\neq D(\mathbf{b})
\]
Then there is at least one element such that
\[
	e\in D(\mathbf{a})-D(\mathbf{b})\quad\mathrm{or}\quad e\in D(\mathbf{b})-D(\mathbf{a})
\]

We claim that $f(\mathbf{a})\neq f(\mathbf{b})$.

Proving by contradiction, if $f(\mathbf{a})=f(\mathbf{b})$, then
\[
	f(\mathbf{a}e)=f(\mathbf{b}e)\quad\forall e\in[n]
\]
If we end up here, we would get the same result for sequences $\mathbf{a}e$ and $\mathbf{b}e$.

Now we consider two situations.

\begin{enumerate}[1.]
	\item $|D(\mathbf{a})|=|D(\mathbf{b})|-1$.
		If so, we select $e=D(\mathbf{a})-D(\mathbf{b})$, then
		\[
			|D(\mathbf{a}e)|=|D(\mathbf{b}e)|-2\neq|D(\mathbf{b}e)|
		\]
		But our algorithm would output same result for $\mathbf{a}e$ and $\mathbf{b}e$.
		
	\item Otherwise.
		We select $e=D(\mathbf{b})-D(\mathbf{a})$, then
		\[
			|D(\mathbf{a}e)|=|D(\mathbf{a})|+1\neq|D(\mathbf{b})|=|D(\mathbf{b}e)|
		\]
		But our algorithm would output same result for $\mathbf{a}e$ and $\mathbf{b}e$.
\end{enumerate}

Both cases lead to a contradiction. Thus
\[
	f(\mathbf{a})\neq f(\mathbf{b})
\]

Now we have proved our claim and we can rethink $f$ as a injection from $2^{[n]}$ (the set of all subsets of $[n]$) to $S$. Naturally we get
\[
	|S|=2^s\geq|2^{[n]}|=2^n
\]
It comes
\[
	s\geq n
\]
So we must have at least $n$ bits to solve the problem.
