\documentclass[12pt,a4]{article}

\usepackage{graphicx,amsmath,amssymb,amsthm, boxedminipage, bm}
\usepackage{float}
\usepackage[body={14.64cm, 24.62cm}, centering, dvipdfm]{geometry}
\usepackage{enumerate}

\newtheorem{theorem}{Theorem}%[section]
\newtheorem{proposition}[theorem]{Proposition}
\newtheorem{lemma}[theorem]{Lemma}
\newtheorem{corollary}[theorem]{Corollary}
\newtheorem{definition}[theorem]{Definition}

\newcommand{\scalar}[2]{\ensuremath{\langle #1, #2\rangle}}
\newcommand{\floor}[1]{\left\lfloor #1 \right\rfloor}
\newcommand{\ceil}[1]{\left\lceil #1 \right\rceil}
\newcommand{\norm}[1]{\|#1\|}
\newcommand{\pfrac}[2]{\left(\frac{#1}{#2}\right)}
\newcommand{\nth}[1]{#1^{\textsuperscript{th}}}


\newcommand{\N}{\mathbb{N}}

\newcommand{\R}{\mathbb{R}}

\newcounter{exercise}

\newcommand{\exercise}{\addtocounter{exercise}{1}\noindent\textbf{Exercise \arabic{section}.\arabic{exercise}. }}



\begin{document}

\date{
	2015-12-06
}

\author{Dumb Ways to Die}

\title{Algorithms and Complexity\\
  Homework Assignment 05
}
\maketitle

\setcounter{section}{
	4
}
\section{Basic Probability Theory}

\exercise{}
Let	$X$ be a random variable taking on values in $\mathbb{N}_0$. Prove that $\mathbb{E}[X] = \sum_{k \geq 1} Pr[X \geq k]$.
	\begin{proof}
		By the definition of expected value, we have
		\begin{eqnarray*}
			\mathbb{E}[X] &= &\sum_{k = 0}^{\infty}kPr[X=k]\\
			&= &\sum_{k = 0}^{\infty}k(Pr[X\geq k] - Pr[X \geq k + 1])\\
			&= &\sum_{k = 0}^{\infty}kPr[X\geq k] - \sum_{k = 1}^{\infty}(k- 1)Pr[X \geq k])\\
			&= &\sum_{k = 1}^{\infty} Pr[X \geq k]
		\end{eqnarray*}
		Thus we've proved $\mathbb{E}[X] = \sum_{k \geq 1} Pr[X \geq k]$, where $k \in \mathbb{N}_0$.
	\end{proof}
\exercise{}
We toss a coin that show $1$ with probability $p$ and $0$ with probability $1-p$. We toss it until it shows $1$. Let $T$ be the number of tosses we have done in total.
\begin{itemize}
	\item What is $Pr[T = k]$?\\
	$T=k$ means we have tossed the coin showing 0 for $(k-1)$times and showing 1 at the $k$-th time. So the probability is
		\begin{eqnarray*}
			Pr[T = k] &= &p(1-p)^{k-1}
		\end{eqnarray*}
	\item What is $Pr[T \geq k]$?\\
	$T \geq k$ means we must have tossed the coin showing 0 for $(k-1)$times. Thus the probability is
		\begin{eqnarray*}
			Pr[T \geq k] &= &(1-p)^{k-1}
		\end{eqnarray*}
	\item What is $\mathbb{E}[T]$?\\
		The expected value of $T$ is
		\begin{eqnarray*}
			\mathbb{E}[T] &= &\sum_{k=1}^{\infty}k Pr[T = k]\\
			&= &\sum_{k = 1}^{\infty} kp(1-p)^{k-1}
		\end{eqnarray*}
		In order to sum it up, we do calculation as following
		\begin{eqnarray*}
			\mathbb{E}[T] - (1 - p)\cdot\mathbb{E}[T] 
			&= &p\cdot\mathbb{E}[T]\\
			&= &\sum_{k = 1}^{\infty} kp(1-p)^{k-1} - \sum_{k = 1}^{\infty} kp(1-p)^{k}\\
			&= &\sum_{k = 1}^{\infty} kp(1-p)^{k-1} - \sum_{k = 1}^{\infty} (k-1)p(1-p)^{k - 1}\\
			&= &\sum_{k = 1}^{\infty} p(1-p)^{k-1}\\
			&= &p\cdot \lim\limits_{k\rightarrow \infty} \frac{1 - (1-p)^k}{1 - (1 - p)} = 1\\
		\end{eqnarray*}
		$\Rightarrow$
		\[\mathbb{E}[T] = 1 / p\]
		Hence the expected value of $T$ is $1/p$.
\end{itemize}

\exercise{}

\exercise{}

\begin{enumerate}
	\item a
	\item b
	\item c
\end{enumerate}

\end{document}
