\documentclass[12pt,a4]{article}

\usepackage{graphicx,amsmath,amssymb,amsthm, boxedminipage, bm}
\usepackage{float}
\usepackage[body={14.64cm, 24.62cm}, centering, dvipdfm]{geometry}
\usepackage{enumerate}

\newtheorem{theorem}{Theorem}%[section]
\newtheorem{proposition}[theorem]{Proposition}
\newtheorem{lemma}[theorem]{Lemma}
\newtheorem{corollary}[theorem]{Corollary}
\newtheorem{definition}[theorem]{Definition}

\newcommand{\scalar}[2]{\ensuremath{\langle #1, #2\rangle}}
\newcommand{\floor}[1]{\left\lfloor #1 \right\rfloor}
\newcommand{\ceil}[1]{\left\lceil #1 \right\rceil}
\newcommand{\norm}[1]{\|#1\|}
\newcommand{\pfrac}[2]{\left(\frac{#1}{#2}\right)}
\newcommand{\nth}[1]{#1^{\textsuperscript{th}}}


\newcommand{\N}{\mathbb{N}}

\newcommand{\R}{\mathbb{R}}

\newcounter{exercise}

\newcommand{\exercise}{\addtocounter{exercise}{1}\noindent\textbf{Exercise \arabic{section}.\arabic{exercise}. }}

\date{2015-12-06}
\author{Dumb Ways to Die}
\title{Algorithms and Complexity\\
	Homework Assignment 05
}


\begin{document}
\maketitle
\setcounter{section}{4}
\section{Basic Probability Theory}

\exercise{}
Let	$X$ be a random variable taking on values in $\mathbb{N}_0$. Prove that $\mathbb{E}[X] = \sum_{k \geq 1} Pr[X \geq k]$.
	\begin{proof}
		By the definition of expected value, we have
		\begin{eqnarray*}
			\mathbb{E}[X] &= &\sum_{k = 0}^{\infty}kPr[X=k]\\
			&= &\sum_{k = 0}^{\infty}k(Pr[X\geq k] - Pr[X \geq k + 1])\\
			&= &\sum_{k = 0}^{\infty}kPr[X\geq k] - \sum_{k = 1}^{\infty}(k- 1)Pr[X \geq k])\\
			&= &\sum_{k = 1}^{\infty} Pr[X \geq k]
		\end{eqnarray*}
		Thus we've proved $\mathbb{E}[X] = \sum_{k \geq 1} Pr[X \geq k]$, where $k \in \mathbb{N}_0$.
	\end{proof}
\exercise{}
We toss a coin that show $1$ with probability $p$ and $0$ with probability $1-p$. We toss it until it shows $1$. Let $T$ be the number of tosses we have done in total.
\begin{itemize}
	\item What is $Pr[T = k]$?\\
	$T=k$ means we have tossed the coin showing 0 for $(k-1)$times and showing 1 at the $k$-th time. So the probability is
		\begin{eqnarray*}
			Pr[T = k] &= &p(1-p)^{k-1}
		\end{eqnarray*}
	\item What is $Pr[T \geq k]$?\\
	$T \geq k$ means we must have tossed the coin showing 0 for $(k-1)$times. Thus the probability is
		\begin{eqnarray*}
			Pr[T \geq k] &= &(1-p)^{k-1}
		\end{eqnarray*}
	\item What is $\mathbb{E}[T]$?\\
		The expected value of $T$ is
		\begin{eqnarray*}
			\mathbb{E}[T] &= &\sum_{k=1}^{\infty}k Pr[T = k]\\
			&= &\sum_{k = 1}^{\infty} kp(1-p)^{k-1}
		\end{eqnarray*}
		In order to sum it up, we do calculation as following
		\begin{eqnarray*}
			\mathbb{E}[T] - (1 - p)\cdot\mathbb{E}[T] 
			&= &p\cdot\mathbb{E}[T]\\
			&= &\sum_{k = 1}^{\infty} kp(1-p)^{k-1} - \sum_{k = 1}^{\infty} kp(1-p)^{k}\\
			&= &\sum_{k = 1}^{\infty} kp(1-p)^{k-1} - \sum_{k = 1}^{\infty} (k-1)p(1-p)^{k - 1}\\
			&= &\sum_{k = 1}^{\infty} p(1-p)^{k-1}\\
			&= &p\cdot \lim\limits_{k\rightarrow \infty} \frac{1 - (1-p)^k}{1 - (1 - p)} = 1\\
		\end{eqnarray*}
		$\Rightarrow$
		\[\mathbb{E}[T] = 1 / p\]
		Hence the expected value of $T$ is $1/p$.
\end{itemize}

\exercise{}

Since there are just $5$ nodes, we denote each vertex by element in $\mathbf{F}_5$.

Define
\[
	\mathbf{E}(\overline{i})\quad\overline{i}\in\mathbf{F}_5
\]
as the expected number of steps the drunk person takes until he reaches $\overline{0}$ if he starts his random walk at vertex $\overline{i}\in\mathbf{F}_5$.

Obviously
\[
	\mathbf{E}(\overline{0})=0
\]
since once he reaches $\overline{0}$ he will never move.

Each step, he walks randomly to one of the two neighbouring vertices. Then
\[
	\mathbf{E}(\overline{i})=\frac{\mathbf{E}(\overline{i+1})+\mathbf{E}(\overline{i-1})}{2}+1
\]

Every equation is linear. We have now a system of linear equations
\[
	\begin{bmatrix}
		1 & & & & \\
		\frac{1}{2} & -1 & \frac{1}{2} & & \\
		 & \frac{1}{2} & -1 & \frac{1}{2} & \\
		 & & \frac{1}{2} & -1 & \frac{1}{2} \\
		\frac{1}{2} & & & \frac{1}{2} & -1
	\end{bmatrix}
	\begin{bmatrix}
		\mathbf{E}(\overline{0}) \\
		\mathbf{E}(\overline{1}) \\
		\mathbf{E}(\overline{2}) \\
		\mathbf{E}(\overline{3}) \\
		\mathbf{E}(\overline{4})
	\end{bmatrix}
	=
	\begin{bmatrix}
		0 \\
		-1 \\
		-1 \\
		-1 \\
		-1
	\end{bmatrix}
\]
Solution of the above system is
\[
	\begin{bmatrix}
		\mathbf{E}(\overline{0}) \\
		\mathbf{E}(\overline{1}) \\
		\mathbf{E}(\overline{2}) \\
		\mathbf{E}(\overline{3}) \\
		\mathbf{E}(\overline{4})
	\end{bmatrix}
	=
	\begin{bmatrix}
		0 \\
		4 \\
		6 \\
		6 \\
		4
	\end{bmatrix}
\]
Done.

\exercise{}

\begin{enumerate}
	\item
	For any $ i\in\F_{p} $, it has 
	\[
		\begin{aligned}
			\Pr[X_{u} = i] &= \sum_{x\in \F_{p}}\Pr[a = x, b = (i - x)u^{-1}] \\
		\end{aligned}
	\]	
	
	Because $ a $, $ b \in \F_{p}$ are independent and uniform at random, we have
	\[ \forall x \in \F_{p}:\Pr[a = x] = \frac{1}{p}, \] \[ \forall y\in\F_{p}:\Pr[b = y] = \frac{1}{p} \] and \[ \Pr[a = x, b = y] = \Pr[a = x]\cdot\Pr[b = y] = \frac{1}{p^{2}}. \]
	
	So \[ \Pr[X_{u} = i] = \sum_{x\in \F_{p}}\Pr[a = x]\Pr[b = (i - a)u^{-1}] = p\cdot\frac{1}{p^{2}} = \frac{1}{p} \]
	
	For any $ u $, $ v\in\F_{p} (u\neq v)$, we have
	\[ 
		\begin{aligned}
			\Pr[X_{u} = i,X_{v} = j] &= \Pr[a + bu = i, a + bv = j] \\
			                         &= \Pr[\left[\begin{matrix}
				                         1 & u \\
				                         1 & v \\
			                         \end{matrix}\right]\left[\begin{matrix}
				                         a \\
				                         b \\
			                         \end{matrix}\right] = \left[\begin{matrix}
				                         i \\
				                         j \\
			                         \end{matrix}\right]]. \\
		\end{aligned}
	\]
	
	Note that
	\[ \det(\left|\begin{matrix}
	1 & u \\
	1 & v \\
	\end{matrix}\right|) = v - u \neq 0. \]
	So the linear system above has a unique solution in $ \F_{p}^{2} $ and denote it as $ (a_{0}, b_{0}) $. 
	Then 
	\[
		\begin{aligned}
			\Pr[X_{u} = i,X_{v} = j] &= \Pr[a = a_{0}, b = b_{0}] \\
			                         &= \frac{1}{p}\cdot\frac{1}{p} \\
			                         &= \Pr[X_{u} = i]\cdot\Pr[X_{v} = j]. \\
		\end{aligned}
	\]
	
	Therefore, the $ X_{u} $ are pairwise independent.
			\item
		
			Firstly, for any distinct $u, v, w \in \mathbb{F}_{p}$
			\begin{eqnarray*}
				&&\text{Pr}[Y_u = i, Y_v = j, Y_w = k] \\
				= &&\text{Pr}[a + bu + cu^2 = i, a + bv + cv^2 = j, a + bw + cw^2 = k] \\
				= &&\text{Pr}[
				\left[
				\begin{matrix}
					1 & u & u^2 \\
					1 & v & v^2 \\
					1 & w & w^2
				\end{matrix}
				\right]
				\left[
				\begin{matrix}
					a \\
					b \\
					c
				\end{matrix}
				\right]
				=
				\left[
				\begin{matrix}
					i \\
					j \\
					k
				\end{matrix}
				\right]
				]
			\end{eqnarray*}
			
			Note that
			\[
				\det
				\left(\left|
				\begin{matrix}
					1 & u & u^2 \\
					1 & v & v^2 \\
					1 & w & w^2
				\end{matrix}
				\right|\right)
				=
				(v - u)(w - u)(w - v)
				\neq
				0
			\]
			
			So the system of linear equations above has and only has one solution in $\mathbb{F}^{3}_{p}$.
			Denote it as $(a_0, b_0, c_0)$.
			
			Therefore,
			\begin{eqnarray*}
				\text{Pr}[Y_u = i, Y_v = j, Y_w = k]
				= \text{Pr}[a = a_0, b = b_0, c = c_0]
				= \frac{1}{p^3}
			\end{eqnarray*}
			
			Secondly, for any distinct $u, v \in \mathbb{F}_{p}$
			\begin{eqnarray*}
				&&\text{Pr}[Y_u = i, Y_v = j] \\
				= &&\text{Pr}[a + bu + cu^2 = i, a + bv + cv^2 = j] \\
				= &&\text{Pr}[
				\left[
				\begin{matrix}
					1 & u & u^2 \\
					1 & v & v^2
				\end{matrix}
				\right]
				\left[
				\begin{matrix}
					a \\
					b
				\end{matrix}
				\right]
				=
				\left[
				\begin{matrix}
					i \\
					j
				\end{matrix}
				\right]
				]
			\end{eqnarray*}
			
			Note that
			\[
				rank
				\left(
				\begin{matrix}
					1 & u & u^2 \\
					1 & v & v^2
				\end{matrix}
				\right) = 2
			\]
			
			So the system of linear equations above has $p$ solutions in $\mathbb{F}^{3}_{p}$.
			
			Therefore,
			\begin{eqnarray*}
				\text{Pr}[Y_u = i, Y_v = j]
				= \frac{p}{p^3}
				= \frac{1}{p^2}
			\end{eqnarray*}
			
			Thirdly,
			\begin{eqnarray*}
				&&\text{Pr}[Y_u = i] \\
				= &&\text{Pr}[a + bu + cu^2 = i]
			\end{eqnarray*}
			
			Obviously, for any $a_0, b_0 \in \mathbb{F}_{p}$, there exists and only exists one $c_0 \in \mathbb{F}_{p}$, such that $a_0 + b_0 u + c_0 u^2 = i$.
			
			Thus, there are altogether $p^2$ solutions.
			
			So
			\begin{eqnarray*}
				&&\text{Pr}[Y_u = i] \\
				= &&\text{Pr}[a + bu + cu^2 = i] \\
				= &&\frac{p^2}{p^3} \\
				= &&\frac{1}{p}
			\end{eqnarray*}
			
			Similarly,
			\begin{eqnarray*}
				&&\text{Pr}[Y_v = j] \\
				= &&\text{Pr}[Y_w = k] \\
				= &&\frac{1}{p}
			\end{eqnarray*}
			
			Therefore,
			\[
				\text{Pr}[Y_u = i] \cdot \text{Pr}[Y_v = j] = \frac{1}{p} \cdot \frac{1}{p} = \frac{1}{p^2}
			\]
			\[
				\text{Pr}[Y_u = i] \cdot \text{Pr}[Y_v = j] \cdot \text{Pr}[Y_w = k] = \frac{1}{p} \cdot \frac{1}{p} \cdot \frac{1}{p} = \frac{1}{p^3}
			\]
			
			Finally, for any distinct $u, v, w \in \mathbb{F}_{p}$
			\[
				\text{Pr}[Y_u = i, Y_v = j] = \text{Pr}[Y_u = i] \cdot \text{Pr}[Y_v = j]
			\]
			\[
				\text{Pr}[Y_u = i, Y_v = j, Y_w = k] = \text{Pr}[Y_u = i] \cdot \text{Pr}[Y_v = j] \cdot \text{Pr}[Y_w = k]
			\]
			
			i.e., $Y_u$ are $3$-wise independent.
		\item
		
			$\mathcal{H} = \{ f | f(x) = a x^2 + b x + c \quad a,b,c \in \mathbb{F}_{p} \}$
			
			\begin{proof}
			
				Obviously, $|\mathcal{H}| = p^3$.
				
				For any distinct $u, v, w \in \mathbb{F}_{p}$ and arbitrary $i, j, k \in \mathbb{F}_{p}$,
				
				\begin{eqnarray*}
					&&\mathop{\text{Pr}} \limits_{h \in \mathcal{H}} [h(u) = i, h(v) = j, h(w) = k] \\
					= &&\mathop{\text{Pr}} \limits_{h \in \mathcal{H}} [a + bu + cu^2 = i, a + bv + cv^2 = j, a + bw + cw^2 = k] \\
					= &&\mathop{\text{Pr}} \limits_{h \in \mathcal{H}}[
					\left[
					\begin{matrix}
						1 & u & u^2 \\
						1 & v & v^2 \\
						1 & w & w^2
					\end{matrix}
					\right]
					\left[
					\begin{matrix}
						a \\
						b \\
						c
					\end{matrix}
					\right]
					=
					\left[
					\begin{matrix}
						i \\
						j \\
						k
					\end{matrix}
					\right]
					]
				\end{eqnarray*}
				
				Note that
				\[
					\det
					\left(\left|
					\begin{matrix}
						1 & u & u^2 \\
						1 & v & v^2 \\
						1 & w & w^2
					\end{matrix}
					\right|\right)
					=
					(v - u)(w - u)(w - v)
					\neq
					0
				\]
				
				So the system of linear equations above has and only has one solution in $\mathbb{F}^{3}_{p}$.
				Denote it as $(a_0, b_0, c_0)$.
				
				Therefore,
				\begin{eqnarray*}
					\mathop{\text{Pr}} \limits_{h \in \mathcal{H}}[h(u) = i, h(v) = j, h(w) = k]
					= \mathop{\text{Pr}} \limits_{h \in \mathcal{H}}[a = a_0, b = b_0, c = c_0]
					= \frac{1}{p^3}
				\end{eqnarray*}
			\end{proof}
		

\end{enumerate}

\end{document}
